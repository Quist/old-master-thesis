\documentclass[UKenglish]{ifimaster}  
\usepackage[utf8]{inputenc}
\usepackage[T1]{fontenc,url}
\usepackage{epigraph}
\urlstyle{sf}
\usepackage{babel,textcomp,csquotes,ifimasterforside,varioref,graphicx}
\usepackage[backend=biber,style=numeric-comp]{biblatex}

\title{Injury surveillance in Sri Lanka}
\subtitle{How to improve data-collection for injury surveillance
using Web-based forms in Sri Lanka}
\author{Henning Grimeland Koller} 

\newcommand{\WHO}{World Health Organization }
\newcommand{\RQ}{How to improve data-collection for injury surveillance
using Web-based forms in Sri Lanka }
\newcommand{\is}{injury surveillance}
\bibliography{mybib}

\begin{document}
\ififorside{}
\frontmatter{}
\maketitle{}

\chapter*{Abstract}
My abstract
\tableofcontents{}
%\listoffigures{}
%\listoftables{}
\mainmatter{}

\chapter{Introduction}
\epigraph{
An injury is the physical damage that results when a human body is suddenly or briefly subjected to intolerable levels of energy.}{\textit{\WHO\cite{who-guide}}}

According to a new document published by \WHO\cite{who-article} 14000 people dies as a result of injuries every day.
Injuries are an important health concern, and a growing problem in some countries. Specifically in developing countries with an increase in economy, motorization and road traffic.
This quick growth in technology and motorization also comes with a great risk of accidents and injury.
Right now road traffic injuries and falls are on rank 9, and 21 respectively on the leading causes of death in 2012.\cite[4]{who-article} 
But because of the growth in many low- and middle-income countries they are expected to rise to rank 7, and 17 by 2030.
And this is just deaths, a small portion of injuries leads to deaths. Most of injuries lead to hospitalization and disability.
To prevent these injuries we need information. We need data telling us why they happened, how they happened and where they happened. 
This will tell us how serious the different injury types are and what can be done to prevent them from happening.

\section*{Injury surveillance}
Injury surveillance is a good way to collect all this data. Surveillance refers to the ongoing and systematic
collection, analysis, interpretation and dissemination of health information.\cite[11]{who-guide}
Basically 

\chapter{Research question}
\textit{"\RQ"}
\\\\
This question has many parts that we need to break down before we can discuss the research method.
First of all \textbf{\is}, as described earlier, is an ongoing and systematic collection of injury information and then creating usable
and understandable data from this information. \textbf{Data-collection} means how all this injury information is collected. These two combined is the "what", the core subject, of the research question.
\textbf{Improving} the data-collection is the outcome, the goal of the research. Improve can mean a lot of things, first and foremost it means speed and quality. 
Speed is important to improve because if the collection process is too long per subject it can hurt the data in several ways. If the process is too long, the data-collector might be tempted to make shortcuts, hurting the quality of the data.
The quantitative side of the data is also hurting if the process is too long, since you will not be able to collect as much as you want.
Speeding up the process also means making it less complex, which again means making it less prone to error. So by making the process faster we're also able to increase the quality of the data.
Quality can further be improved just by digitizing the process. This will remove many factors that can hurt the quality, like bad handwriting and environmental damage.
\textbf{Web-based form} is the boundary of the research question. The data-collection is to be improved by web-based forms.

\section*{Research method}
My method

\chapter{Conclusion}

\backmatter{}
\printbibliography
\end{document}
