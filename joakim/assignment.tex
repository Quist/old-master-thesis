\documentclass[UKenglish, 12pt]{article}
\usepackage[utf8]{inputenc} 
\usepackage[T1]{fontenc,url}
\urlstyle{sf}
\usepackage{babel,textcomp,csquotes,ifikompendiumforside,varioref,graphicx}
\usepackage[backend=bibtex,sorting=none]{biblatex}

\title{Improving Injury Surveillance in low and middle income countries}
\subtitle{Research Design}
\author{Joakim Johanson Lindquister} 

\bibliography{injuries}

\begin{document}
\ififorside{}

\begin{abstract}
As a part of the Action Research Workshop I will in this paper discuss my research question for my master thesis and how I plan to perform the research supporting it. The selected research method of my thesis is a reasearch method called Action Research. I will first give an introduction to my problem area, Injury Surveillance, before I elaborate on our research design and in specific our research method.
\end{abstract}

\section*{Introduction} 
% State research problem motivating the research (one sentence)
Every year, over 5 million humans are killed of some sort of injury and many more are permanently injured\cite[page~12]{who-guide}. This represents a major global public health issue. Traditionally, injuries were looked upon as simple accidents or random events which simply could't be prevented. However, today we know that in a society we can introduce guidelines, laws, technical advancements or awareness campaigns which are proven measures for preventing injuries. Example of such measures that helps prevents damanges, are the use of seat belts and flame-resistant clothing.

However, in order for health authories to apply effective, cost-effective measurements, most countries needs better information. They need to know what are the frequent reasons for people dying and the circumenstances behind those injures. Especially in low and middle income countries, resources are limited and the authorities must prioritize. Health services are generally much less accessible in these countries and the death rates from injuries are much higher than in wealthier countries.

To look into this problem area and see how the injury surveillance process can be improved, we have selected Sri lanka as an case study for the injury surveillance process in low and middle income countries. Sri Lanka is a south-asian country located on a island south of India with over 21 million inhabitants\cite{snl-sri-lanka}. It has a history of religious differences and conflicts, and a long-lasting civil war between the tamils and sengalesers ended in 2009. Sri Lanka is a developing country with an uneven distribution of wealth. The relationship between Sri Lanka and Norway is good as a result of years of development cooperation aswell as Norway being involved in the peace process\cite{udep-sri-lanka}. \\

My research partner and I are investigating how the injury surveillance is done today in Sri Lanka. Specifically are we looking into the process at the National Hospital in Colombo where they have deployed an instance of DHIS2, which they want to utilize for injury surveillance.


\section*{Research question}

\emph{How to improve data-collection for injury surveillance using web-based forms in low/middle income countries}
\\ \\
Defining a research question is the starting point when undertaking academic research. The question should be clear and focused and will act as an guide when undertaking the research question. My partner and I have written a draft for our research question which can be subject to change as we work with our research. The question brings up some new terms that I will discuss here.

\paragraph*{Injury Surveillance}
Injury surveillance refers to the ongoing and systematic collection, analysis, interpretation and dissemination of health information\cite{who-guide}. What information should be collected and how it is collected varies from country to country, but the World Health Organizatino has created guidelines which provides recomenneded data all injury surveillance systems should collect.
\paragraph*{Improvement of data-collection} 
Improvment of the exist injury surveillance process is the goal of our research. But how what isimprovement and how can it be measure it to see if we made improvment to the process? We must keep in mind the overall goal of injury surveillance which is to collect statisticts so that the health authories and goverment can issue appropriate measures. Therefore it is clear that an improvement must be to better the quality of injury surveillence statistics. To understand how we can improve the statistics we neeed to understand the situation in Sri Lanka today. How many of injuries are reported, and what is the data quality of the data actually reported? \\
Talking to people at the hospital makes it clear that the issue today is that the existing data-collection tool, a paper form, is very tideous for the nurses to use. The main issuse is that it takes alot of time to fill out one form, and the data clerks does not have enough time to cover all the patients. By speeding up the time used per patient, data empirical data for the injury surveillance is improved.

\paragraph*{Data collection} 
It is important to discuss and document the data collection process. How is today and how is it data-collection related to WHO guidelines. In our research we're looking into the process in the National Hospital in Sri Lanka. Today the data is collected after care at the wards in the accident unit. The data is then entered into DHIS2.

\paragraph*{Web-based forms}
Web-based forms means that the data-collection are done by entering information into a form on a web-page. We are devloping an application to fit into an existing implementation of DHIS2.

\paragraph*{Low/Middle income countries}
We are looking into countries with limited health resources and we're using Sri Lanka as an case study.
 

 \section*{The research process}
 The research process is an iterative process and the research question can change as we emphasizes different parts of the problem area. The main phases of the reasearch phases are
 \begin{enumerate}
 \item \textbf{Research review:} Look into existing relvant studies and theories.
 \item \textbf{Research design:} About organizing the research activity and how the research data should be collected.
 \item \textbf{Data collection:} Observe events and data.
 \item \textbf{Data analysis:} Analyse the collected data.
 \item \textbf{Discussion:} Discuss the data collected and how it fits with the research goal.
 \end{enumerate}

\section*{Research design}
The input to the research design phase is the research question along with the accociated theory. The research design gives direction and systemizes the research.

\subsection*{Research method - Action Research}
To find improvements to the injury surveillance process we plan to apply a practial reasearch method called action reasearch. The goal of action reasearch is to solve a particular problem and produce guidelines for best practice\cite{denscombe}. In action research, the researcher participates in the practiallity of performing a change to the process while simultaneously conduction research.

%Discuss why we choose this method. 
As a part of our thesis, we will develop a new web-based system for injury surveillance data-collection and participate in the process where the nurses starts using it. In order for us to understand how to improve the process, it is very important for us to learn more about the current sitation. We have no previous experience from health informatics and how the systems work in Sri Lanka, so it is critical for us to learn more from the actual users of the system.  What do they think of it? What problems are they experiencing?

Since we are creating a new system and implementing it, action research is a relvant research method for us. As we implement the new system, we can simultaniously conduct reseach.


\subsubsection*{Action Research Cycle}
Action research is an iterative process which involves different phases that are reported.

\subsubsection*{Diagnosing} In this phase we identifiy the problems and their underlying causes. To do this, it is very important to understand the context of the project. Why is this project neccessary? Who are the stakeholders? Furthermore,  we create an hypothesis of the research phenomenon to be used in later phases of the action research cycle. \\

In our project we are working together with the people supervising the injury surveillance project in Sri Lanka. In this ongoing process we are collecting information about how the injury surveillance are done today and how we can improve it. 

\subsubsection*{Action planning} is the proposal of solutions that can improve the problem situation. Our propsal is that implementing a intuitive web-based solution would improve the surveillance data-collection.

\subsubsection*{Intervention} In this phase, the proposed solution from the action planning is implemented. Altough not planned in detail yet, we plan to build a working prototype as soon as possible. We will then ask the team in Sri Lanka to deploy it and test. With the feedback from this phase, we will alter the system and contine working on it. This will be an interative process where we hope to involve the team in Sri Lanka as much as possible. We also have plans on traveling down to Sri Lanka to obsere the process in details.

\subsubsection*{Evaluation} Evalation of the interventation phase. Did the steps taken in the intervention phase improve the problems discovered in the diagnosis phase? \\
In this phase we will together with the Injury Surveillance team evalaute the implementation of the Injury Surveillane project. Did it improve the data-collected speed?
\subsubsection*{Specifying learning} This phase includes documenting the learning outcomes the action research cycle. 

\pagebreak
\printbibliography
\end{document}