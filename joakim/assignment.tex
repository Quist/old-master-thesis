\documentclass[UKenglish, 12pt]{article}
\usepackage[utf8]{inputenc} 
\usepackage[T1]{fontenc,url}
\urlstyle{sf}
\usepackage{babel,textcomp,csquotes,ifikompendiumforside,varioref,graphicx}
\usepackage[backend=biber,style=numeric]{biblatex}

\title{Improving Injury Surveillance in low/middle income countries}
\subtitle{Research Design}
\author{Joakim Johanson Lindquister} 
\bibliography{injuries}

\begin{document}
\ififorside{}

\begin{abstract}
As a part of the Action Research Workshop I will in this paper discuss my research question for my master thesis and how I plan to perform the research supporting it. The selected research method of my thesis is a reasearch method called Action Research. I will first give an introduction to my problem area, Injury Surveillance, before I elaborate on our research design and in specific our research method.
\end{abstract}

\section*{Introduction} 
% State research problem motivating the research (one sentence)
Every year, over 5 million humans are killed of some sort of injury and many more are permanently injured\cite[site~12]{who-guide} which represents a major global public health issue. Traditionally, injuries were looked upon as simple accidents or random events which simply could't be prevented. However, today we know that in a society we can introduce guidelines, laws, technical advancements or awareness campaigns which are proven measures for preventing injuries. Example of such measures that helps prevents damanges, are the use of seat belts and flame-resistant clothing.

However, in order for health authories to apply effective, cost-effective measurements, most countries needs better information. They need to know what are the frequent reasons for people dying and the circumenstances behind those injures. Especially in low and middle income countries, resources are limited and the authorities must priorities. Health services are generally much less accessible in these countries and the death rates are much higher than in wealthier countries.

To look into this problem area and see how the injury surveillance process can be improved, we have selected Sri lanka as an case study for the injury surveillance process in low and middle income countries. Sri Lanka is a south-asian country located on a island south of India. 
% Si noe om norsk bistand idag og situasjonen nå. 

\section*{Research question}
% 2. State your main research question addressing the problem (one sentence)
\emph{How to improve data-collection for injury surveillance using web-based forms in low/middle income countries}
\\ \\
The research question needs to be discussed in detail. What does improve really mean? Using Sri Lanka as a case study \\

\paragraph*{Injury Surveillance}
Define injury surveillance. Referer to WHO and so on.

\paragraph*{Improvement of data-collection} 
What is the improvement and how can  we measure it. We must keep in mind the overall goal of injury surveillance which is to collect statisticts so that the health authories and goverment can issue tiltak. Therefore it is clear that an improvement must be to better the quality of injury surveillence statistics. To understand how we can improve the statistics we neeed to understand the situation today. How many of injuries are reported, and what is the data quality of the data actually reported? \\
Talking to people at the hospital makes it clear that the issue today is that the existing data-collection tool, a paper form, is very tideous for the nurses to use. The main issuse is that it takes alot of time to fill out one form, and the data clerks does not have enough time to cover all the patients. By speeding up the time used per patient, data empirical data for the injury surveillance is improved.

\paragraph*{Web-based forms}

\paragraph*{Low/Middle income countries}
Define them. compare to norway. Talk about Sri Lanka as a case study
 

 \section*{The research process}
 The research process is an interative process and the research question can change as I emphasizes different parts of the problem area. The main phases of the reasearch phases are
 \begin{enumerate}
 \item \textbf{Research review:} Look into existing relvant studies and theories.
 \item \textbf{Research design:} About organizing the research activity, how the research data should be collected.
 \item \textbf{Data collection:} Observe events and data.
 \item \textbf{Data analysis:} Analyse the collected data
 \item \textbf{Discussion:} Discuss

 \end{enumerate}


\section*{Research design}
The input to the research design phase is the research question along with theory.

\subsection*{Key Concepts}
The key concepts of our research is injury surveillance and process accoicated with it.

\subsection*{Research method}
\subsubsection*{Action Research}
To find improvements to the injury surveillance process we plan to apply a practial reasearch method, called action reasearch. The goal of action reasearch is to solve a particular problem and produce guidelines for best practice(Denscombe 2010, p.6 - wiki). In action research the researcher participates in the practiallity of performing a change to the process while simultaneously conduction research.

%Discuss why we choose this method. 
As a part of our thesis, we will develop a new web-based system for injury surveillance data-collection and participate in the process where the nurses starts using it. We will obsere and interview. Iterative process. Deploy early and get feedback from the actual users of the system. It is difficult for us to understand the situation for the people actually using the system. Travel down there and. Observe the hospital workers use the system. What do they think of it? What problems and feedback are they having? Measure the speed before/after. Improve the system while we're there, iterations.

\paragraph{Action Research Cycle}
Action research is an iterate process which involves different phases that are reported.
\begin{description}
\item\textbf{{Diagnosing}} In this phase we identifiy the problems and their underlying causes. We create an hypothesis of the research phenomenon to be used in later phases of the action research cycle. 
\item{\textbf{Action planning}} is the proposal of solutions that can improve the problem situation. Our propsal is that implementing a intuitive web-based solution would improve the surveillance data-collection.
\item{\textbf{Intervention}} In this phase the propsed solution from the action planning is implemented.
\item{\textbf{Evaluation}} Evalation of the interventation phase. Did the steps taken in the intervention phase improve the problems discovered in the diagnosis phase?
\item{\textbf{Specifying learning}} This phase includes documenting the learning outcomes the action research cycle. 

\end{description}

\subsection{Level of investigation}
Our research question dsxc. However, it would be very difficult to study the implementation and improvement of injury surveillance in all low and middle income countries. Therefor we have choosen to select injury surveillance process as it is implemented in the national hospital in Sri Lanka. In our thesis we will use Sri Lanka as an case study. The problem of this aproach is if the lessons learned in Sri Lanka can be generalized to other low/middle income countries.

\subsection{Plan for intervention}

\pagebreak
\printbibliography
\end{document}