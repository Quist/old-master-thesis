\documentclass[UKenglish]{article}
\usepackage[utf8]{inputenc} 
\usepackage[T1]{fontenc,url}
\urlstyle{sf}
\usepackage{babel,textcomp,csquotes,ifikompendiumforside,varioref,graphicx}
\usepackage[backend=biber,style=numeric-comp]{biblatex}

\title{Improving Injury Surveillance in Sri Lanka}
\subtitle{Research Design}
\author{Joakim Johanson Lindquister} 


\begin{document}
\ififorside{}
\begin{abstract}
Your abstract goes here...
...
\end{abstract}
\section*{Introduction} 
Injury Surveillance is the axcasd. Så skriver jeg kort intro til surveillance i Sri Lanka idag. 
\\ \\
Sri Lanka is a south-asian country located on a island south of India. Diverse country with asdasdxczxd dasd asd asd asd sad asd asd. Konflikt -> Norsk fredsarbeid. Politiske situasjonen idag

Norsk bistand idag og situasjonen nå. 


\section*{Research Problem}
1. State your main research problem motivating the research (one sentence)
2. State your main research question addressing the problem (one sentence)

\section*{Research method}
The research method is the strategy of how to find empirical data about the world. Something about GI leaning towards the phenomenological paradigm.
\subsection{Ressearch goal}
How should we measure if the improvement was successfull?
3. Define the key concept used in the thesis (one sentence)

4. State your unit of analysis and empirical setting (one sentence)

5. State your research approach/method (one sentence)

6. State your main (expected) contribution (one sentence)

7. Write a title that communicates the essence of the research

\subsection{Action Research}
To find improvements to the injury surveillance process we plan to apply a practial reasearch method, called action reasearch. The goal of action reasearch is to solve a particular problem and produce guidelines for best practice(Denscombe 2010, p.6 - wiki). In action research the researcher participates in the practiallity of performing a change to the process while simultaneously conduction research.

In our thesis we will develop a new electronoic sumission system for injury surveillance data and participate in the process where the nurses starts using it. We will obsere and interview. Iterative process. Deploy early and get feedback from the actual users of the system. It is difficult for us to understand the situation for the people actually using the system. Travel down there and. Obsere the hospital workers use the system. What do they think of it? What probelms and feedback are they having? Measure the speed before/after. Improve the system while we're there, iterations.



\end{document}